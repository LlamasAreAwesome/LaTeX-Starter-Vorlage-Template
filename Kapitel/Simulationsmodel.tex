\chapter{Simulationsmodel}
Vorstellung des Models -> Berechnung nach Grundlagen Kap.

Vessel L =110 m - B = 11,4 m - T =2,8 m - \(\Delta\)= 3 140 T - Deadweight: 
2300 T, 
Lightweight: 840 T - Installed power : 1000 kW
Lwl=109,240m Cb=0,8942 Cm=0,9980 Cp=0,8960 Cwp=0,9387 Lcb\%=0,745 App=27,31m2 
(1+k2)=1,50 \(v_{max}\) = 15,2 km/h

Nozzle 19A KA 4-70 (2) 2 Diam 1,59 m 4 blades p/D=1,190 Ae/Ao=0,700    -   
Sm=1707m2 At=0,00m2 Abt=0,00m2 Hb=0,00m2

Waterway : W =156,00 m   w =120,00 m    h=4,50 m Vcr = 16,4km/h 
\(\sqrt{gh}\)23,9km/h 93\%Vcr = 15,2km/h

Hm=3,98m   Bc=138,0m       Bi=198<->225m   Ac=621,0m2   Ab=31,9m2   
Bc/B=12,11   h/T=1,61   m=Ab/Ac=0,051   Ac/Ab=19,5

\begin{table}[htbp]
	\caption{Vessel dimensions -> 	Dand \& Ferguson, Romisch (wenn B/T 
	abgerundet)}
	\centering
	\begin{tabular}{c c c c c }
		& CB &B/T&L/B&L/T \\
		Beispiel Vessel & 0,8942 & 4,0714 & 9,6491 & 39,2857 \\
	
	\end{tabular}
\end{table}
\begin{table}[htbp]
	\caption{Waterway \& interface vessel - Waterway -> Romisch, Huuska-Guliev, 
	Yoshimura- Ohtsu, Eryuzlu 2}
	\centering
	\begin{tabular}{c c p{3cm} c c c }
		& h/T &\(A_B/A_c\)&\(h_T/h\) &L/h & Bc/B(a) \\
		Beispiel Vessel & 1.868 & (B*h)/(h*(W+w)/2) = 31.92 /628.6 = 0.0508& -  
		& 39.28 & Ac/h / B = 19,69\\
		
		
	\end{tabular}
\end{table}